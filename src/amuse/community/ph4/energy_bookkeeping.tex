\documentclass[12pt]{article}
\pagestyle{empty}
\usepackage[margin=1in]{geometry}

\newcommand{\half}{{\ensuremath\textstyle\frac12}}
\newcommand{\subsub}[2]{{\ensuremath{#1}_{_{{#2}}}}}

\begin{document}

\section*{Energy Bookkeeping in the AMUSE Multiples Module}

The basic sequence of events in managing a multiple interaction is as
follows:

\begin{enumerate}

\item A pair of stars is identified as interacting in the top-level
  ``gravity'' dynamical code.  Optionally, the multiples module may
  include other top-level neighbors in the interaction.

\item All top-level stars involved are removed from the gravity code.
  If indices $I$ and $J$ count top-level objects in the interaction
  and $K$ refers to other top-level objects, then the energy removed
  from the gravity code is
  $$
	\Delta E_{rem} = E_0 + \phi_{rem},
  $$
  where
  $$
	E_0 = \half\sum_I\,\subsub{m}{I}\subsub{v}{I}^2
		 ~+~ \sum_I\sum_{J>I}\subsub{\phi}{IJ}
  $$
  is the total internal energy of the interacting subsystem,
  $$
	\subsub{\phi}{IJ} = -\frac{G\subsub{m}{I}\subsub{m}{J}}
				  {\subsub{r}{IJ}}\,,
  $$
  and
  $$
	\phi_{rem} = -\sum_I\sum_K\,\subsub{\phi}{IK}
  $$
  is the external potential on the subsystem due to the rest of the
  top-level stars.  The interacting stars are subsequently integrated
  as an isolated small-N system.

\item The subsystem is given substructure by restoring the binary
  information stored in the multiples data structure.  Let $E_{mul,I}$
  be the total internal energy associated with the initial top-level
  object $I$.  Then, when the internal structure is reinstated, the
  new energy of the interacting subsystem is
  $$
	E_1 = E_0 + \sum_I\,E_{mul,I} + \Delta\phi_1,
  $$
  where
  $$
	\Delta\phi_1 = \sum_I\sum_{i\in I}\sum_{j\in J>I}\,
						\subsub{\phi}{ij}
			- \sum_I\sum_{J>I}\subsub{\phi}{IJ}
  $$
  is the tidal potential energy change associated with resolving all
  multiples.  Here, indices $i$ and $j$ range over ``leaf'' stars in
  the subsystem, and the notation $i\in I$ means that leaf $i$ is part
  of top-level node $I$.  Note that $\Delta\phi_1$ includes only
  potentials between leaves in different top-level nodes.  Potentials
  between leaves in the same node $I$ are included in $E_{mul,I}$.  In
  a scattering experiment, $\Delta\phi_1$ is minimized by expanding
  the top-level system to ``scattering'' distance before reinstating
  the binary substructure.  We do not calculate it directly in
  practice, although it useful to retain it for purposes of
  discussion.

\item We run the interacting subsystem to completion as a scattering
  experiment.  The energy of the subsystem at the end is
  $$
	E_2 = E_1 + \Delta E_{int},
  $$
  where $\Delta E_{int}$ is the (normally small) integration error
  associated with the small-N integrator.

\item We then reclaim and remove all stable multiples, leaving us with
  a new top-level energy of the interacting subsystem
  $$
	E_3 = E_2 - \sum_F\,E_{mul,F} - \Delta\phi_2,
  $$
  where $E_{mul,F}$ is as defined above, with the sum now running over
  the top-level particles in the final configuration, and
  $\Delta\phi_2$ is defined analogously to $\Delta\phi_1$.

\item The top-level system is then rescaled at constant energy $E_3$
  back into a volume comparable to that of the initial interaction.
  Binary nodes that are too large are removed and corrected for at
  this point, modifying $E_3$, $E_{mul}$, and $\Delta\phi_2$.

\item Finally, the top-level nodes are reinserted into the gravity
  module.  The total energy added is
  $$
	\Delta E_{ins} = E_3 + \phi_{ins},
  $$
  where $\phi_{ins}$ is defined analogously to $\phi_{rem}$.

\end{enumerate}

Thus the total energy change in the top-level system due to the entire
sequence of events just outlined is
\begin{eqnarray*}
	\Delta E_{top} &=& \Delta E_{ins} - \Delta E_{rem} \\
		      &=& \phi_{ins} - \phi_{rem}
				+ \sum_I\,E_{mul,I} - \sum_F\,E_{mul,F}
 				+ \Delta\phi_1 - \Delta\phi_2
				+ \Delta E_{int} \\
		      &=& \Delta\phi_{top} - \Delta E_{mul} 
 				+ \Delta\phi_1 - \Delta\phi_2
				+ \Delta E_{int},
\end{eqnarray*}
so, in terms of the total energies associated with the gravity and
small-N modules,
$$
	\Delta\left(E_{top}+E_{mul}\right)
		= \Delta\phi_{top} + \Delta\phi_1 - \Delta\phi_2
				  + \Delta E_{int},
$$
where all the terms on the right-hand side are tidal corrections or
integration errors and are (in principle) small.  We currently adopt
the strategy of simply accumulating and monitoring the sum of the
tidal terms, as
$$
	\phi_{tidal} = \sum_{encounters}\left(\Delta\phi_{top}
					    + \Delta\phi_1
					    - \Delta\phi_2\right),
$$
in which case the quantity 
$$
	E_{top} + E_{mul} - \phi_{tidal}
$$
should be conserved, up to integration errors in the top-level and
small-N modules.

Assuming that all bookkeeping is done correctly, we can determine the
internal tidal terms from
$$
	\Delta\phi_1 - \Delta\phi_2 = E_3 - E_0 + \Delta E_{mul}
						- \Delta E_{int}.
$$
We monitor this quantity to ensure that it is small, as expected.

\end{document}
